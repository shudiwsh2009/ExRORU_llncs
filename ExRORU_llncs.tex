\documentclass[dvips,...]{llncs}
\usepackage{makeidx}
\usepackage{enumerate}
\usepackage[dvips]{graphicx}
\usepackage{subfigure}
\usepackage{authblk}
\usepackage{amssymb}
\usepackage{amsmath}
\usepackage{amsfonts}
\usepackage{booktabs}
\usepackage{threeparttable}
\usepackage{algorithm}
\usepackage{algorithmicx}
\usepackage{algpseudocode}
\usepackage{textcomp}
\usepackage{multirow}
\usepackage{diagbox}
\usepackage[usenames,dvipsnames]{color}
\usepackage{xcolor}
\usepackage{colortbl}
\definecolor{gray1}{gray}{.9}
\definecolor{gray2}{gray}{.8}
\definecolor{gray3}{gray}{.7}
\renewcommand{\algorithmicrequire}{\textbf{Input}}
\renewcommand{\algorithmicensure}{\textbf{Output}}
\renewcommand{\textfraction}{0.15} 
\renewcommand{\topfraction}{0.85} 
\renewcommand{\bottomfraction}{0.65} 
\renewcommand{\floatpagefraction}{0.60}
\makeatletter 
  \newcommand\figcaption{\def\@captype{figure}\caption} 
  \newcommand\tabcaption{\def\@captype{table}\caption} 
  \newcommand{\tabincell}[2]{\begin{tabular}{@{}#1@{}}#2\end{tabular}}
\makeatother
\setlength{\abovecaptionskip}{4pt} 
\setlength{\belowcaptionskip}{4pt}
\setlength{\floatsep}{4pt}
\setlength{\textfloatsep}{4pt}
\setlength{\intextsep}{4pt}
\begin{document}
\frontmatter 
\pagestyle{headings}
\addtocmark{ExRORU: Extended Refined Ordering Relations with Uncertainty}

\mainmatter
\title{ExRORU: a Method for Behavioral Differentiation Detection between Process Models}
\titlerunning{ExRORU: Extended Refined Ordering Relations with Uncertainty}

\author[$1$]{Shuhao Wang}
\author[$1$]{Lijie Wen}
\author[$1$]{Jianmin Wang}
\author[$2$]{Jianwen Su}
\author[$3$]{Akhil Kumar}
\authorrunning{Shuhao Wang et al.}
\tocauthor{Shuhao Wang, Lijie Wen, Jianmin Wang, Jianwen Su, and Akhil Kumar}
\affil[$1$]{School of Software, Tsinghua University, Beijing 100084, P.R. China \authorcr
\texttt{shudiwsh2009@gmail.com,\{wenlj,jimwang\}@tsinghua.edu.cn}}
\affil[$2$]{Department of Computer Science, UC Santa Barbara, USA \authorcr
\texttt{su@cs.ucsb.edu}}
\affil[$3$]{Smeal College of Business, Penn State University, University Park, \authorcr
State College, PA 16802, USA \authorcr
\texttt{akhilkumar@psu.edu}}
\institute{}

\maketitle

\begin{abstract}
Tao Jin has proposed new ordering relations with uncertainty between execution of tasks in acyclic process models. He also gave an algorithm to compute the refined ordering relations based on unfolding technology. However, his algorithm cannot work well for cyclic WF-nets and process models with silent transitions. We extend his work and present a refinement of the relations for cyclic process models named ExRORU, short for extended refined ordering relations with uncertainty. To better measure the differentiation between process models, we introduce the notion of sequential direct adjacency. We utilize them to detect the behavioral differentiation between process models. Experiments on real-life models show that compared to existing algorithms, our measure is both effective and efficient.
\keywords{Process Model, Refined Ordering Relations with Uncertainty, Sequential Direct Adjacency, Behavioral Differentiation Detection}
\end{abstract}

\section{Introduction}\label{sec:introduction}

In a seminal paper \cite{jin2014computing}, Tao Jin has proposed new ordering relations with uncertainty between execution of tasks in acyclic process models. He refines the causal and concurrency relations between two tasks in a concurrent system with uncertainty according to whether one task is always executed with the other task in the same instance. He then proposes some rules for adjacent tasks and transitive laws for non-adjacent tasks, based on which he proposes an algorithm to compute the refined ordering relations for acyclic process models. The algorithm is done on the unfolding of a WF-net \cite{mcmillan1995technique,esparza1996improvement}. Although Tao Jin'work is elegant, it has some drawbacks. We illustrate the situations where his work fails through the following examples.

\begin{figure}[htbp]
	\begin{minipage}[b]{0.48\textwidth}
		\centering
		\subfigure[] {
			\centering
			\includegraphics[width=1.0\textwidth]{fig_sda_example_1}
			\label{fig:sdaExampleA}
		}
		\subfigure[] {
			\centering
			\includegraphics[width=1.0\textwidth]{fig_sda_example_2}
			\label{fig:sdaExampleB}
		}
		\caption{Two WF-nets with silent transitions\label{fig:sdaExample}}
	\end{minipage}%
	\hfill
	\begin{minipage}[b]{0.48\textwidth}
		\centering
		\subfigure[] {
			\centering
			\includegraphics[width=1.0\textwidth]{fig_nfc_example_1}
			\label{fig:nfcExampleA}
		}
		\subfigure[] {
			\centering
			\includegraphics[width=1.0\textwidth]{fig_nfc_example_2}
			\label{fig:nfcExampleB}
		}
		\caption{WF-nets with and without non-free-choice constructs\label{fig:nfcExample}}
	\end{minipage}
\end{figure}

\begin{example}\label{ex:drawback1}
The model of Figure \ref{fig:examplePetri} is a Petri net with a cycle (through transition ``E''). According to Jin'paper, his algorithm cannot work well with cyclic Petri nets. The problem arises from the Global Fairness Assumption \cite{kindler1999liveness}.
\end{example}

\begin{example}\label{ex:drawback2}
Further, silent transitions have not been taken into consideration so that two models with different behavioral sementics may be regarded as the same in Jin's work. For example, the full firing sequences of the model in Figure \ref{fig:sdaExampleA} are $\langle I,A,B,O\rangle$, $\langle I,B,A,O\rangle$, $\langle I,A,O\rangle$ and $\langle I,B,O\rangle$, while the full firing sequences of the model in Figure \ref{fig:sdaExampleB} are $\langle I,A,B,O\rangle$, $\langle I,B,A,O\rangle$, $\langle I,A,O\rangle$, $\langle I,B,O\rangle$ and $\langle I,O\rangle$. But Jin'work cannot distinguish them from each other.
\end{example}

\begin{example}\label{ex:drawback3}
In a model with non-free-choice constructs \cite{wen2007mining}, sequential relations do not necessarily satisfy transitivity. Take the models in Figure \ref{fig:nfcExample} for example, we can capture the causal relations between $A$ and $C$ as well as $C$ and $E$ from both models. According to Jin's work, it can be infered that $A$ and $E$ satisfy the causal relation. However, this isn't correct in the model with non-free-choice constructs of Figure \ref{fig:nfcExampleA} because only $C$ and $D$ can be executed if $A$ has been executed in the first place.
\end{example}

To solve these problems, we extend his ordering relations to more trivial cases which is inspired by the work of DecSerFlow \cite{van2006decserflow}, and introduce the concept of sequential direct adjacency to better differentiate the behavioral semantics of two process models. Furthermore, since our refined relations have a strong ability to represent the behavioral semantics of a process model, even when they contain special constructs such as silent transitions or non-free-choice, we can utilize them as a proper method to detect the differentiation between two process models.

Differentiation detection can be widely used in many scenarios and it can be treated as the basis of a proper similarity measure between process models. There have been various proposals to topics about differentiation detection and similarity measure based on textual information, the topology of process models, or their execution semantics \cite{weidlich2011efficient}. We give a quick introduction to some of them related to our work.

Zha et al. \cite{zha2010workflow} represent a model's behavior by transition adjacency relations (TAR for short). They use the Jaccard coefficient of TAR sets to measure the similarity between models. But TAR cannot tackle the long-distance dependences between transitions, thus it is not able to handle specific structures like non-free-choice constructs. BP algorithm \cite{weidlich2011efficient} represents the behavioral semantics of a model based on behaviroal profiles between transitions, but it cannot deal with silent transitions and distinguish parallel and loop structures. PTS algorithm by Professor Wang \cite{wang2010behavioral} and its improvement CFS by Dong et al. \cite{dong2014cfs} utilize refined trace sets of models to compute their similarity and are the most realistic results so far. But extracting the trace sets by both their methods are very time-consuming. Polyvyanyy et al. \cite{polyvyanyy20144c} propose another definition of relations between transitions named 4C spectrum but the hunderds of types of relations in 4C are too complex for comprehension and hard to compute. It is not a practical method in differentiation detection.

In this paper, we propose a new method called ExRORU to represent the relations between transitions in a process model, which has better precision than the ones in Jin's work. Further, we introduce the notion of sequential direct adjacency to make our method even more precise. Using ExRORU and SDA, we can detect the behavioral differentiation between process models. Our method is much more effective than other methods ultilizing relations between transitions and efficient than those which should capture the huge trace sets.

The remainder of this paper is structured as follows. The next section gives preliminaries that will be used in this paper. The concept and computation of extended refined ordering relations as well as sequential direct adjacency are introduced in Section \ref{sec:relations}. Section \ref{sec:experiments} shows the results of an experimental evaluation, before we conclude the paper and give an outlook on the future work in Section \ref{sec:conclusion}.

\section{Preliminaries}\label{sec:preliminaries}
Since Petri net is suitable to analyze the behavior of process models, we will show our algorithm on Petri net. The concept of the Petri net has its origin in Carl Adam Petri's dissertation \cite{petri1966kommunikation}.

\begin{definition}[Petri net]\label{def:petrinet}
A Petri net is a triple $(P,T,F)$, where $P$ is a finite set of places, $T$ is a finite set of transitions ($P\cap T=\emptyset$ and $P\cup T=\emptyset$), and $F\subseteq(P\times T)\cup(T\times P)$ is a set of arcs (flow relation).
\end{definition}

We use $M:P\rightarrow\mathbb{N}$ to denote a marking of Petri net $N$, where $\mathbb{N}$ is the set of natural numbers. A Petri net system is a pair $\Sigma=(N,M_{0})$, where $N$ is a Petri net and $M_{0}$ is the initial marking of $N$. For more details about Petri net, please refer to the work of Murata \cite{murata1989petri}. A Petri net which models a workflow is called a WorkFlow net (WF-net) \cite{van1998application}.

\begin{definition}[WF-net]\label{def:wfnet}
A Petri net $PN=(P,T,F)$ is a WF-net (Workflow net) if and only if $PN$ has two special places: $i$ and $o$, place $i$ is a source place: $\bullet i=\emptyset$ and place $o$ is a sink place: $o\bullet =\emptyset$. If we add a transition $t^{*}$ to $PN$ which connects place $o$ with $i$ (i.e. $\bullet t^{*}=\{o\}$ and $t^{*}\bullet=\{i\}$), then the resulting Petri net is strongly connected.
\end{definition}

We extract the relations between transitions in a WF-net. An intuitive idea is to generate the reachability graph of a model. We need to consider all possible interleavings of concurrent transitions, which causes the state explosion problem in a concurrent system \cite{mcmillan1995technique}. Javier Esparza proposed the notion of complete prefix unfolding to avoid this \cite{esparza1996improvement}. Therefore, we use his work instead.

\begin{definition}[Ordering Relations]\label{def:orderingRelations}
There are three types of ordering relations between nodes of a net,
	\begin{itemize}
		\item[-] two nodes $x$ and $y$ are in causal relation, denoted by $x<y$, if the net contains a path with at least one arc leading from $x$ to $y$;
		\item[-] $x$ and $y$ are in conflict relation, denoted by $x\#y$, if the net contains two disjoint paths $s\cdot t_{1}...x_{1}$ and $s\cdot t_{2}...x_{2}$ starting at the same place $s$;
		\item[-] $x$ and $y$ are in concurrency relation, denoted by $x~co~y$, if neither $x<y$ nor $y<x$ nor $x\#y$.
	\end{itemize}
\end{definition}

\begin{definition}[Occurrence net]\label{def:occurrenceNet}
An occurrence net is a Petri net $O=(B,E,A)$ such that $\forall x,y\in B\cup E:(x,y)
\in A^{+}\Rightarrow(y,x)\notin A^{+}$ and $\forall b\in B:|\bullet b|\leq 1$.
\end{definition}

\begin{definition}[Branching process]\label{def:branchingProcess}
A branching process of a Petri net system $\Sigma=(P,T,F,M_{0})$ is a labelled occurrence net $\beta=(B,E,A,f)$ where the labelling function $f$ satisfies the following properties:
	\begin{itemize}
		\item[-] $f(B)\subseteq P$ and $f(E)\subseteq T$ ($f$ preserves the nature of nodes);
		\item[-] for every $e\in E$, the restriction of $f$ to $\bullet e$ is a bijection between $\bullet e$ (in $\Sigma$) and $\bullet f(e)$ (in $\beta$), and similarly for $e\bullet$ and $f(e)\bullet$ ($f$ preserves the environments of transitions);
		\item[-] the restriction of $f$ to $Min(O)$ is bijection between $Min(O)$ and $M_{0}$ ($\beta$ ``starts'' at $M_{0}$), where $Min(O)$ denotes the set of minimal elements of $B\cup E$ with respect to the causal relation;
		\item[-] for every $e_{1},e_{2}\in E$, if $\bullet e_{1}=\bullet e_{2}$ and $f(e_{1})=f(e_{2})$ then $e_{1}=e_{2}$ ($\beta$ does not duplicate the transitions of $\Sigma$).
	\end{itemize}
\end{definition}

There are only sequential and parallel structures but neither loop nor exclusive structures in a branching process. A Petri net system may have infinite branching processes, which differ on ``how much it unfolds''. The complete prefix unfolding is the minimal one which contains all the markings contained in the branching process.

\begin{definition}[Complete Prefix Unfolding]\label{def:cpu}
Let $O=(B,E,A,f)$ be a occurrence net and $e\in E$ be any event.
	\begin{itemize}
		\item[-] A local configuration $\lceil e\rceil$ of an event $e$ in an occurrence net is the set of events that precede $e$.
		\item[-] The final marking of a local configuration $Mark(\lceil e\rceil)$ is the set of conditions that are marked after all the events in $\lceil e\rceil$ fire.
		\item[-] An adequate order $\prec$ is a strict well-founded partial order on local configurations, so that $\lceil e\rceil\subset\lceil e'\rceil$ implies $\lceil e\rceil\prec\lceil e'\rceil$\footnote{Several definitions of \textit{adequate order} exist, and we use the total order for 1-safe systems defined in \cite{esparza1996improvement}.}.
		\item[-] An event $e$ of a branching process is a cutoff event if there exists a corresponding event $e'$, such that $Mark(\lceil e\rceil)=Mark(\lceil e'\rceil)$ and $\lceil e'\rceil\prec\lceil e\rceil$.
		\item[-] A complete prefix unfolding is the greatest backward closed subnet of an occurrence net containing no events after cutoff events.
	\end{itemize}
\end{definition}

Definitions about complete prefix unfolding and related concepts come from the work of Esparza \cite{esparza1996improvement} and Polyvyanyy \cite{polyvyanyy2010structuring}.

\begin{figure}[htbp]
\centering
\subfigure[A sound WF-net] {
	\centering
	\includegraphics[width=0.45\textwidth]{fig_example_petri}
	\label{fig:examplePetri}
}
\subfigure[The complete prefix unfolding of \subref{fig:examplePetri}] {
	\centering
	\includegraphics[width=0.45\textwidth]{fig_example_cpu}
	\label{fig:exampleCpu}
}
\caption{A sound WF-net example and its complete prefix unfolding\label{fig:examplePetriAndCpu}}
\end{figure}

\begin{example}\label{ex:petriAndCpu}
Figure \ref{fig:examplePetriAndCpu} shows a sound WF-net and its complete prefix unfolding. In the unfolding, events labelled as $B$ and $E$ are cutoff events.
\end{example}

Our extended relations are inspired by the work in \cite{van2006decserflow}, which utilizes the relations between activities to model a process.
\begin{definition}[Relation Formulas]\label{def:relationFormulas}
Let $A$ and $B$ be two activities of a process, $A$ and $B$ are in the following relation formulas\footnote{Note that the relation formulas are not complete here, since we only need part of them. For more details, please refer to the work of van der Alast \cite{van2006decserflow}.}:
	\begin{itemize}
		\item[-] Response: when activity $A$ executes, activity $B$ has to be executed after it.
		\item[-] Precedence: activity $B$ could not have been executed until activity $A$ was executed.
		\item[-] Alternate Response: after the execution of an activity $A$, activity $B$ has to be executed and between the execution of each two $A$ at least one activity $B$ has to be executed.
		\item[-] Alternate Precedence: before the execution of an activity $B$, activity $A$ has to be executed first and between the xecution of each two $B$ at least one activity $A$ has to be executed.
	\end{itemize}
\end{definition}

\begin{example}\label{ex:relationFormulas}
In the model of Figure \ref{fig:examplePetri}, $D,E$ in the execution sequence $\langle I,D,E,$\\$A,D,C,E,D,O\rangle$ satisfy the formula \textit{alternate response} and \textit{alternate precedence} while transitions $D,O$ do not satisfy the formula \textit{alternate response} and transitions $I,D$ do not satisfy the formula \textit{alternate precedence}.
\end{example}

\section{Extended Refined Ordering Relations with Uncertainty}\label{sec:relations}
In this section, we introduce the concept of extended refined ordering relations with uncertainty and its calculation from a WF-net. Let $\Sigma=(P,T,F,M_{0})$ be a sound WF-net, and $U=(B,E,A,f)$ be $\Sigma$'s complete prefix unfolding. Let $x,y\in E$ be two events of the unfolding $U$.

\subsection{Extended Refined Causal Relations with Uncertainty}\label{subsec:causalAndInverseCausal}
Complete prefix unfolding use the technique of cutoff event to avoid state explosion problem. We can set a mapping between cutoff conditions (conditions after cutoff events) and the conditions after the corresponding event of a cutoff event. Therefore, the branching process expressed in a complete prefix unfolding never ends on a cutoff condition, but will continue from the mapping of a cutoff condition instead.

As mentioned before, a branching process only contains sequential and parallel structures. We use expressions like $[ab\{c,d\}e]$ to represent the semantics of a branching process. Successive letters denote the sequential structure while letters in a brace mean a parallel structure. Different branches of a parallel structure are separated by a comma and could be nested. For example, in the complete prefix unfolding of Figure \ref{fig:exampleCpu}, $[I\{D,AC\}O]$, $[I\{D,BC\}O]$ and $[I\{DED,AC\}O]$ are some branching processes.

Let $\Theta$ be the set containing all the branching process of $U$, and $\Theta_{x}$ be the set of the branching processes containing event $x$, i.e, $\Theta_{x}=\{\beta\in\Theta|x\in\beta\}$. Our extended causal relations are defined using the notion of $\Theta_{x}$.

\begin{definition}[Always Causal]\label{def:alwaysCausal}
Events $x$ and $y$ are in always causal relation (denoted as $x\twoheadrightarrow y$) iff: $\forall\beta_{x}\in\Theta_{x}$, $x$ and $y$ are in \textit{alternate response} relation.
\end{definition}

\begin{definition}[Never Causal]\label{def:neverCausal}
Events $x$ and $y$ are in never causal relation (denoted as $x\nrightarrow y$) iff: $\forall\beta_{x}\in\Theta_{x}$, $x$ and $y$ are not in \textit{response} relation.
\end{definition}

\begin{definition}[Sometimes Causal]\label{def:sometimesCausal}
Events $x$ and $y$ are in sometimes causal relation (denoted as $x\rightharpoonup y$) iff: neither $x\twoheadrightarrow y$ nor $x\nrightarrow y$.
\end{definition}

\begin{example}\label{ex:causalRelation}
In the model of Figure \ref{fig:exampleCpu}, $A$ and $C$ are in \textit{always causal} relation ($A\twoheadrightarrow C$) since they are in \textit{alternate response} relation among all the branching processes containing $A$. $D$ and $O$ are in \textit{sometimes causal} relation ($D\rightharpoonup O$) since there may be another $D$ between the execution of the first $D$ and $O$, indicating that $D$ and $O$ are not in \textit{alternate response} relation. $A$ and $B$ are in \textit{never causal} relation ($A\nrightarrow B$).
\end{example}

With the help of \textit{alternate precedence} and \textit{precedence} relations, \textbf{Always Inverse Causal} ($\twoheadleftarrow$), \textbf{Sometimes Inverse Causal} ($\leftharpoonup$) and \textbf{Never Inverse Causal} ($\nleftarrow$) can be defined in a similar way to the three causal relations.

% \begin{definition}[Always Inverse Causal]\label{def:alwaysInverseCausal}
% Events $y$ and $x$ are in always inverse causal relation (denoted as $y\twoheadleftarrow x$) iff: $\forall\beta_{y}\in\Theta_{y}$, $x$ and $y$ are in \textit{alternate precedence} relation, i.e., among all the branching processes containing event $y$, every instance of $y$ has to be preceded by an instance of $x$ and the next instance of event $y$ cannot be executed before the next instance of event $x$ is executed.
% \end{definition}

% \begin{definition}[Never Inverse Causal]\label{def:neverInverseCausal}
% Events $y$ and $x$ are in never inverse causal relation (denoted as $y\nleftarrow x$) iff: $\forall\beta_{y}\in\Theta_{y}$, $x$ and $y$ are not in \textit{precedence} relation, i.e., among all the branching processes containing event $y$, before the execution of every $y$, there mustn't be an instance of $a$ executed.
% \end{definition}

% \begin{definition}[Sometimes Inverse Causal]\label{def:sometimesInverseCausal}
% Events $y$ and $x$ are in sometimes inverse causal relation (denoted as $y\leftharpoonup x$) iff: neither $y\twoheadleftarrow x$ nor $y\nleftarrow x$.
% \end{definition}

\begin{example}
In the model of Figure \ref{fig:examplePetri}, $O$ and $D$ are in \textit{always inverse causal} relation ($O\twoheadleftarrow D$) since they are in \textit{alternate precedence} relation among all the branching processes containing $O$. $C$ and $A$ are in \textit{sometimes inverse causal} relation ($C\leftharpoonup A$) since there may be no instance of transition $A$ in some branching processes containing $C$ (such as $[I\{D,BC\}O]$). $A$ and $B$ are in \textit{never inverse causal} relation ($A\nleftarrow B$).
\end{example}

The relations above are collectively called causal relation with uncertainty (denoted as $\rightarrow$) and inverse causal relation with uncertainty (denoted as $\leftarrow $). We have identified several cases of causal and inverse causal relations and turn them into abstract formulas, as shown in Figure \ref{fig:causalCases}. In thses cases, $A$ and $B$ are events of an unfolding. The edge labeled with ``Loop $X$'' means that there is a path starting from some condition after $X$ and ending in some condition before $X$ so that $X$ can be executed more than once. The edge labeled with ``Skip $X$'' means that there is a path starting from some condition before $X$ and ending in some condition after $X$ so that $X$ may not be executed. %In other words, events inside a \textit{loop} structure can be executed more than once while events inside a \textit{skip} structure may not be executed.

\begin{figure}[htbp]
\centering
\subfigure[] {
	\begin{minipage}[b]{0.45\textwidth}
		\centering
		\includegraphics[width=0.6\textwidth]{fig_causal_case_a_1}\\
		\includegraphics[width=0.6\textwidth]{fig_causal_case_a_2}
	\end{minipage}
	\label{fig:causalCaseA}
}
\subfigure[] {
	\begin{minipage}[b]{0.45\textwidth}
		\centering
		\includegraphics[width=0.6\textwidth]{fig_causal_case_b_1}
		\includegraphics[width=0.6\textwidth]{fig_causal_case_b_2}
	\end{minipage}
	\label{fig:causalCaseB}
}
\subfigure[] {
	\begin{minipage}[b]{0.45\textwidth}
		\centering
		\includegraphics[width=0.8\textwidth]{fig_causal_case_c}
	\end{minipage}
	\label{fig:causalCaseC}
}
\subfigure[] {
	\begin{minipage}[b]{0.45\textwidth}
		\centering
		\includegraphics[width=0.8\textwidth]{fig_causal_case_d}
	\end{minipage}
	\label{fig:causalCaseD}
}
\caption{Abstract formulas of causal and inverse causal relations. \subref{fig:causalCaseA} $A\rightharpoonup B, B\twoheadleftarrow A$; \subref{fig:causalCaseB} $A\twoheadrightarrow B, B\leftharpoonup A$; \subref{fig:causalCaseC} $A\twoheadrightarrow B, B\twoheadleftarrow A$; \subref{fig:causalCaseD} $A\rightharpoonup B, B\leftharpoonup A$.\label{fig:causalCases}}
\end{figure}

% \textcolor{red}{Compress the following two paragraphs}
For convenience, we use symbol $*$ to represent any numbers of sequential events excluding $A$ and $B$. In Figure \ref{fig:causalCaseA}, we have branching processes such as $[A*B],[A*A*B],[A*A*A*B]...$ for the upper unfolding and $[A*B],[A*]$ for the lower unfolding, both of which indicate that $A$ and $B$ are in \textit{sometimes causal} relation and \textit{always inverse causal} relation, i.e., $A\rightharpoonup B,B\twoheadleftarrow A$. Similarly in Figure \ref{fig:causalCaseB}, we have $A\twoheadrightarrow B,B\leftharpoonup A$.

In Figure \ref{fig:causalCaseC}, a loop structure in the middle part of a branching process will certainly not affect the extended relations between $A$ and $B$, neither will an exclusive structure, i.e., $A\twoheadrightarrow B,B\twoheadleftarrow A$. However, if there are exclusive structures across both $A$ and $B$, such as the unfolding in Figure \ref{fig:causalCaseD}, which has branching processes such as $[A*B],[A*],[*B],[*]$, then $A\rightharpoonup B$ and $B\leftharpoonup A$.

\subsection{Extended Refined Concurrent Relations with Uncertainty}\label{subsec:concurrent}
A trace, or a full firing sequence is a finite sequence of events $\sigma\in E^{*}$, leading from the source state to the end state by executing the events in order. Let $\Omega$ be the set containing all the traces of $U$, and $\Omega_{x}$ be the set of the traces containing event $x$, i.e., $\Omega_{x}=\{\sigma\in\Omega|x\in\sigma\}$. Our extended concurrent relations are defined using the notion of $\Omega_{x}$.

We use $\sigma\uparrow X$ to denote the projection of $\sigma$ onto some event set $X\subseteq E$, i.e., a trace $\sigma'\subseteq\sigma$ which only contains those events in $X$ and remains their order in $\sigma$. Let $P(\sigma,x,y)=\sigma\uparrow\{x,y\}$ be the projection of $\sigma$ onto events $x$ and $y$.

\begin{definition}[Always Concurrent]\label{def:alwaysConcurrent}
$x$ and $y$ are in always concurrent relation (denoted as $x\Updownarrow y$) iff: $\forall\sigma_{x}\in\Omega_{x}$,
	\begin{itemize}
		\item[-] $x$ and $y$ are in concurrency relation, i.e., $x~co~y$;
		\item[-] $|P(\sigma_{x},x,y)|\%2=0\wedge\forall_{0\leq k<|P|/2}((P(\sigma_{x},x,y)_{2k+1}=x\wedge P(\sigma_{x},x,y)_{2k+2}=y)\vee(P(\sigma_{x},x,y)_{2k+1}=y\wedge P(\sigma_{x},x,y)_{2k+2}=x)\vee(P(\sigma_{x},x,y)_{2k+1}=y\wedge P(\sigma_{x},x,y)_{2k+2}=y))$.
	\end{itemize}
\end{definition}

\begin{definition}[Never Concurrent]\label{def:neverConcurrent}
$x$ and $y$ are in never concurrent relation (denoted as $x\nparallel y$) iff: $\forall\sigma_{x}\in\Omega_{x}$, $x$ and $y$ are not in concurrency relation.
\end{definition}

\begin{definition}[Sometimes Concurrent]\label{def:sometimesConcurrent}
$x$ and $y$ are in sometimes concurrent relation (denoted as $x\Uparrow y$) iff: neither $x\Updownarrow y$ nor $x\nparallel y$.
\end{definition}

The relations above are collectively called concurrent relations with uncertainty (denoted as $\parallel$). We have also identified several cases of concurrent relations and turn them into abstract formulas, as shown in Figure \ref{fig:concurrentCases}. In these formulas, events $A$ and $B$ are all in \textit{concurrency} relation.

\begin{figure}[htbp]
\centering
\subfigure[] {
	\begin{minipage}[b]{1\textwidth}
		\centering
		\includegraphics[width=0.24\textwidth]{fig_concurrent_case_a_1}%
		\hspace{0.5in}%
		\includegraphics[width=0.24\textwidth]{fig_concurrent_case_a_2}
	\end{minipage}
	\label{fig:concurrentCaseA}
}
\subfigure[] {
	\begin{minipage}[b]{0.3\textwidth}
		\centering
		\includegraphics[width=0.9\textwidth]{fig_concurrent_case_b}
	\end{minipage}
	\label{fig:concurrentCaseB}
}
\subfigure[] {
	\begin{minipage}[b]{0.3\textwidth}
		\centering
		\includegraphics[width=0.9\textwidth]{fig_concurrent_case_c}
	\end{minipage}
	\label{fig:concurrentCaseC}
}
\subfigure[] {
	\begin{minipage}[b]{0.3\textwidth}
		\centering
		\includegraphics[width=0.9\textwidth]{fig_concurrent_case_d}
	\end{minipage}
	\label{fig:concurrentCaseD}
}
\caption{Abstract formulas of concurrent relations. \subref{fig:concurrentCaseA} $A\Uparrow B,B\Updownarrow A$; \subref{fig:concurrentCaseB} $A\Updownarrow B,B\Updownarrow A$; \subref{fig:concurrentCaseC} $A\Updownarrow B,B\Updownarrow A$; \subref{fig:concurrentCaseD} $A\Uparrow B,B\Updownarrow A$.\label{fig:concurrentCases}}
\end{figure}

In Figure \ref{fig:concurrentCaseA}, we have traces such as $\langle I*A*B*\rangle ,\langle I*B*A*\rangle ,\langle I*A*\rangle $ for the left unfolding and traces such as $\langle I*A*B*\rangle ,\langle I*A*A*B*\rangle ,\langle I*A*A*A*B*\rangle ...$ for the right unfolding, which do not satisfy the second condition in Definition \ref{def:alwaysConcurrent}. Therefore, events $A$ and $B$ of the unfoldings in Figure \ref{fig:concurrentCaseA} are in \textit{sometimes concurrent} relation, i.e., $A\Uparrow B$. As for the unfolding in Figure \ref{fig:concurrentCaseB}, events $A$ and $B$ may be executed both once or not executed at all, indicating that $A\Updownarrow B$. It is easily seen that traces in the unfolding of Figure \ref{fig:concurrentCaseC} satisfy the second rule in Definition \ref{def:alwaysConcurrent}, so $A$ and $B$ are in \textit{always concurrent} relation, i.e., $A\Updownarrow B$. The unfolding in Figure \ref{fig:concurrentCaseD} shows a special structure called non-free choice construct, which we will go into details later.

\subsection{Computing ExRORU Based on Unfolding}\label{subsec:computationOfRelations}
%As mentioned before, an intuitive idea to derive our relations is based on reachability graphs. Such technique would face a state explosion problem, especially when the WF-net contains a parallel with large scale of transitions. Theoretically, the reachability graph will derive all the reachable states by combining the executions of transitions within a parallel structure. Another problem is that we actually cannot distinguish the causal relation caused by loop structure and parallel structure using the reachability technique. By definitions, causal relations caused by parallel structure are not considered in our extended refined ordering relations. 
As mentioned before, using reachability graphs to derive our relations would face a state explosion problem. And we actually cannot distinguish the causal relation caused by loop and parallel structure using the reachability technique, the latter of which is not considered in our definitions. So we adopt the unfolding technology in this paper. The reason why we do not derive the relations based on sound WF-net itself is that a WF-net describes the structure of a process rather than behavior, which is just the focus of unfolding technique.

Another intuitive idea on the unfolding technology is to check our relations based on all the branching processes, i.e., $\Theta$ of a complete prefix unfolding. However, $\Theta$ may be an infinite set, which is not practical to compute and utilize. Therefore, motivated by the abstract formulas in Figure \ref{fig:causalCases} and Figure \ref{fig:concurrentCases}, we will derive our relations on a complete prefix unfolding by tracing the directed path from one event to another and check every condition to determine whether there is a \textit{loop} or \textit{skip} structure.

We may notice the fact that the computations of causal relation and inverse causal relation are similar in the way that we regard inverse causal relation as causal relation in the reversion of the unfolding. In another word, if we reverse the direction of every edge in the unfolding, then the causal relations in the new unfolding (actually the new graph is not an unfolding, but we use the saying to explain our computation here) are the same as the inverse causal relations in the original unfolding. This indicates that we can use the algorithm of causal relations to derive the inverse causal relations by backtracing nodes along the paths of an unfolding.

As for the extended concurrent relations between events $a$ and $b$, there are actually two relations which should be taken into consideration, i.e., $a\parallel b$ and $b\parallel a$. We firstly find the gate of the parallel structure, and trace the path from the gate to $a$ and $b$. By checking every condition to find out all the \textit{loop} and \textit{skip} structure, we can derive the extended concurrent relations between two events.

More useful information are the extended relations between transitions in the WF-net, which can be utilized to check the consistency or measure the similarity between process models. We focus on the computation of our relations on a WF-net based on its unfolding.

There may be more than one corresponding event (shadow event in \cite{wang2013efficient}) in the complete prefix unfolding related to a transition in the original WF-net. These corresponding events are different from each other and are treated as unique events when we derive the relations between events in the unfolding. Therefore, when we try to derive the ExRORU between transitions in the WF-net, we need to check the relations between each pair of corresponding events of those transitions. Based on this, we give the computation of ExRORU between transitions. Let $Corr_{E}(A)$ be the set containing all the corresponding events of transition $A$.

\begin{definition}[Extended Causal Relations Between Transitions]\label{def:causalRelations}
Transitions $A$ and $B$ are in
	\begin{itemize}
		\item[-] always causal relation (denoted as $A\twoheadrightarrow B$) iff: $\forall a\in Corr_{E}(A), \exists b\in Corr_{E}(B),\\s.t.~a\twoheadrightarrow b$;
		\item[-] never causal relation (denoted as $A\nrightarrow B$) iff: $\forall a\in Corr_{E}(A), b\in Corr_{E}(B),\\a\nrightarrow b$;
		\item[-] sometimes causal relation (denoted as $A\rightharpoonup B$) iff: neither $A\twoheadrightarrow B$ nor $A\nrightarrow B$.
	\end{itemize}
\end{definition}

\begin{definition}[Extended Inverse Causal Relations Between Transitions]\label{def:inverseCausalRelations}
Transitions $B$ and $A$ are in
	\begin{itemize}
		\item[-] always inverse causal relation (denoted as $B\twoheadleftarrow A$) iff: $\forall b\in Corr_{E}(B), \exists a\in Corr_{E}(A), s.t.~b\twoheadleftarrow a$;
		\item[-] never inverse causal relation (denoted as $B\nleftarrow A$) iff: $\forall b\in Corr_{E}(B), a\in Corr_{E}(A), b\nleftarrow a$;
		\item[-] sometimes inverse causal relation (denoted as $B\leftharpoonup A$) iff: neither $B\twoheadleftarrow A$ nor $B\nleftarrow A$.
	\end{itemize}
\end{definition}

\begin{definition}[Extended Concurrent Relations Between Transitions]\label{def:concurrentRelations}
Transitions $A$ and $B$ are in
	\begin{itemize}
		\item[-] always concurrent relation (denoted as $A\Updownarrow B$) iff: $\forall a\in Corr_{E}(A),\exists b\in Corr_{E}(B),s.t.~a\Updownarrow b$;
		\item[-] never concurrent relation (denoted as $A\nparallel B$) iff: $\forall a\in Corr_{E}(A),b\in Corr_{E}(B),a\nparallel b$;
		\item[-] sometimes concurrent relation (denoted as $A\Uparrow B$) iff: neither $A\Updownarrow B$ nor $A\nparallel B$.
	\end{itemize}
\end{definition}

\begin{example}\label{ex:examplePetriRelations}
Using the notions and computations described in the previous sections, we can derive ExRORU of the model in Figure \ref{fig:examplePetri}. For convenience, we use a matrix to represent the three types of relations, as shown in Table \ref{tab:example_relations}. Transitions in rows are the former ones in a relation while transitions in columns are the latter ones in a relation. For example, the second row are the relations between $A$ and the other transitions. Due to limited space, we show three types of relations in one table cell. The rightmost table cell with relation symbols represents the relations between $A$ and $O$, i.e., $A\twoheadrightarrow O,A\nleftarrow O,A\nparallel O$.
\end{example}

\begin{table}[htbp]
\caption{Extended relations of the model in Figure \ref{fig:examplePetri}\label{tab:example_relations}}
\centering
	\begin{tabular}{c|c|c|c|c|c|c|c} \hline
		Transitions & $A$ & $B$ & $C$ & $D$ & $E$ & $I$ & $O$\\ \hline
		$A$ 
			& $\nrightarrow\nleftarrow\nparallel$
			& $\nrightarrow\nleftarrow\nparallel$
			& $\twoheadrightarrow\nleftarrow\nparallel$
			& $\nrightarrow\nleftarrow\Updownarrow$
			& $\nrightarrow\nleftarrow\Uparrow$
			& $\nrightarrow\twoheadleftarrow\nparallel$
			& $\twoheadrightarrow\nleftarrow\nparallel$
			\\ \hline
		$B$ 
			& $\nrightarrow\nleftarrow\nparallel$
			& $\nrightarrow\nleftarrow\nparallel$
			& $\twoheadrightarrow\nleftarrow\nparallel$
			& $\nrightarrow\nleftarrow\Updownarrow$
			& $\nrightarrow\nleftarrow\Uparrow$
			& $\nrightarrow\twoheadleftarrow\nparallel$
			& $\twoheadrightarrow\nleftarrow\nparallel$
			\\ \hline
		$C$ 
			& $\nrightarrow\leftharpoonup\nparallel$
			& $\nrightarrow\leftharpoonup\nparallel$
			& $\nrightarrow\nleftarrow\nparallel$
			& $\nrightarrow\nleftarrow\Updownarrow$
			& $\nrightarrow\nleftarrow\Uparrow$
			& $\nrightarrow\twoheadleftarrow\nparallel$
			& $\twoheadrightarrow\nleftarrow\nparallel$
			\\ \hline
		$D$ 
			& $\nrightarrow\nleftarrow\Uparrow$
			& $\nrightarrow\nleftarrow\Uparrow$
			& $\nrightarrow\nleftarrow\Uparrow$
			& $\rightharpoonup\leftharpoonup\nparallel$
			& $\rightharpoonup\leftharpoonup\nparallel$
			& $\nrightarrow\leftharpoonup\nparallel$
			& $\rightharpoonup\nleftarrow\nparallel$
			\\ \hline
		$E$ 
			& $\nrightarrow\nleftarrow\Uparrow$
			& $\nrightarrow\nleftarrow\Uparrow$
			& $\nrightarrow\nleftarrow\Uparrow$
			& $\twoheadrightarrow\twoheadleftarrow\nparallel$
			& $\rightharpoonup\leftharpoonup\nparallel$
			& $\nrightarrow\leftharpoonup\nparallel$
			& $\rightharpoonup\nleftarrow\nparallel$
			\\ \hline
		$I$ 
			& $\rightharpoonup\nleftarrow\nparallel$
			& $\rightharpoonup\nleftarrow\nparallel$
			& $\twoheadrightarrow\nleftarrow\nparallel$
			& $\twoheadrightarrow\nleftarrow\nparallel$
			& $\rightharpoonup\nleftarrow\nparallel$
			& $\nrightarrow\nleftarrow\nparallel$
			& $\twoheadrightarrow\nleftarrow\nparallel$
			\\ \hline
		$O$ 
			& $\nrightarrow\leftharpoonup\nparallel$
			& $\nrightarrow\leftharpoonup\nparallel$
			& $\nrightarrow\twoheadleftarrow\nparallel$
			& $\nrightarrow\twoheadleftarrow\nparallel$
			& $\nrightarrow\leftharpoonup\nparallel$
			& $\nrightarrow\twoheadleftarrow\nparallel$
			& $\nrightarrow\nleftarrow\nparallel$
			\\ \hline
	\end{tabular}
\end{table}

\subsection{Sequential Direct Adjacency}\label{subsec:sda}
We can distinguish most process models from each other by their executing semantics using ExRORU. However, there are still models with different semantics but the same ExRORU, especially when silent transitions are introduced. Silent transitions do not appear in any log trace \cite{de2003workflow}. For more details about silent transitions and their mining, please refer to the work of Wen \cite{wen2010mining}.

\begin{table}[htbp]
\centering
\tabcaption{ExRORU of the two models in Figure \ref{fig:sdaExample}\label{tab:sdaExample}}
\begin{tabular}{c|c|c|c|c} \hline
	Transitions & $A$ & $B$ & $I$ & $O$\\ \hline
	$A$
		& $\nrightarrow\nleftarrow\nparallel$
		& $\nrightarrow\nleftarrow\Uparrow$
		& $\nrightarrow\twoheadleftarrow\nparallel$
		& $\twoheadrightarrow\nleftarrow\nparallel$
		\\ \hline
	$B$
		& $\nrightarrow\nleftarrow\Uparrow$
		& $\nrightarrow\nleftarrow\nparallel$
		& $\nrightarrow\twoheadleftarrow\nparallel$
		& $\twoheadrightarrow\nleftarrow\nparallel$
		\\ \hline
	$I$
		& $\rightharpoonup\nleftarrow\nparallel$
		& $\rightharpoonup\nleftarrow\nparallel$
		& $\nrightarrow\nleftarrow\nparallel$
		& $\twoheadrightarrow\nleftarrow\nparallel$
		\\ \hline
	$O$
		& $\nrightarrow\leftharpoonup\nparallel$
		& $\nrightarrow\leftharpoonup\nparallel$
		& $\nrightarrow\twoheadleftarrow\nparallel$
		& $\nrightarrow\nleftarrow\nparallel$
		\\ \hline
\end{tabular}
\end{table}

As shown in Table \ref{tab:sdaExample}, ExRORU of models in Figure \ref{fig:sdaExample} are the same, but the behavioral sementics of them are not (refer to Example \ref{ex:drawback2}). By looking into the executing semantics of these models, we find that the introduction of silent transitions influences the adjacency of visible transitions, i.e., $I$ can be directly followed by $O$ in Figure \ref{fig:sdaExampleB}, but not in Figure \ref{fig:sdaExampleA}. Inspired by the idea of TAR \cite{zha2010workflow}, we propose the concept of sequential direct adjacency to solve the problem of silent transitions.

\begin{definition}[Sequential Direct Adjacency]\label{def:sda}
Let $\Sigma=(P,T,F,M_{0})$ be a WF-net and $A,B\in T$ are two visible transitions. $A$ and $B$ are in sequential direct adjacency iff:
	\begin{itemize}
		\item[-] $A$ and $B$ are not in cocurrency relation;
		\item[-] $\exists\sigma=\langle t_{1},t_{2},...,t_{n}\rangle\in\Omega,1\leq i<j\leq n,t_{i}=A,t_{j}=B,s.t.~\forall k\in(i,j),t_{k}$ is an silent transition.
	\end{itemize}
$A\rightarrow_{D}B$ (``D" for \textit{direct}) if they are in sequential direct adjacency while $A\rightarrow_{I}B$ (``I" for \textit{indirect}) if not. Similarly, we can also define SDA on $\leftarrow$ relation, denoted as $B\leftarrow_{D}A$. The set of all the transition pairs which are in sequential direct adjacency is the SDA-set of a model.
\end{definition}

We use sequential direct adjacency to distinguish those transition pairs which can be executed adjacently from those which cannot. By applying Definition \ref{def:sda} on the two models in Figure \ref{fig:sdaExample}, we derive the \textit{SDA-set} as follows:
\begin{displaymath}
	\begin{aligned}
		\text{(a)} & A\rightarrow_{D}O, B\rightarrow_{D}O, I\rightarrow_{D}A, I\rightarrow_{D}B\\
		& O\leftarrow_{D}A, O\leftarrow_{D}B, A\leftarrow_{D}I, B\leftarrow_{D}I\\
		\text{(b)} & A\rightarrow_{D}O, B\rightarrow_{D}O, I\rightarrow_{D}A, I\rightarrow_{D}B, I\rightarrow_{D}O\\
		& O\leftarrow_{D}A, O\leftarrow_{D}B, A\leftarrow_{D}I, B\leftarrow_{D}I, O\leftarrow_{D}I
	\end{aligned}
\end{displaymath}

Using the combination of ExRORU and SDA, we can distinguish the two models with silent transitions as well as many similar model pairs from each other efficiently.

% \section{Similarity Measure}\label{sec:similarity}
% In this section, we introduce how extened relations and sequential direct adjacency are used to measure the similarity between two process models.

% By combining extended relations and sequential direct adjacency together, we give the computation of the similarity between two WF-nets. For any ordered transition pair $A,B$ in a process model $M$, we can capture three types of extended relations between them, i.e., $R_{\rightarrow}^{M}(A,B)$, $R_{\leftarrow}^{M}(A,B)$ and $R_{\parallel}^{M}(A,B)$. We use a couple $R_{h}^{M}(A,B)=(\gamma,\delta)$ to represent their extended relations, where $h\in\{\rightarrow,\leftarrow,\parallel\}$, $\gamma$ is the uncertainty (``A'' for \textit{always}, ``S'' for \textit{sometimes} and ``N'' for \textit{never}), $\delta$ is the sequential direct adjacency (``D'' for \textit{direct} if $A\rightarrow_{D}B\in SDA-set$ and ``I'' for \textit{indirect} otherwise). For convenience, we use ${R_{h}^{M}}_{pair}$ to denote the transition pair in $R_{h}^{M}$.

% On the other hand, we think two extended relations with the same $\delta$ are more similar than two extended relation with different $\delta$, so we set a weight $\lambda$ to balance the influence of sequential direct adjacency.

% For two extended relations $R_{h}^{P}=(\gamma_{P},\delta_{P})$ and $R_{h}^{Q}=(\gamma_{Q},\delta_{Q})$ with the same $h$ and transition pair from two models $P$ and $Q$ respectively, we have:
% \begin{equation}\label{eq:relationOperations}
% 	\begin{aligned}
% 		R_{h}^{P}\cap R_{h}^{Q}&=
% 			\begin{cases}
% 				0 & \gamma_{P}\neq\gamma_{Q}\\
% 				\lambda & \gamma_{P}=\gamma_{Q}\wedge\delta_{P}\neq\delta_{Q}\\
% 				1 & \gamma_{P}=\gamma_{Q}\wedge\delta_{P}=\delta_{Q}
% 			\end{cases}\\
% 		R_{h}^{P}\cup R_{h}^{Q}&=1
% 	\end{aligned}
% \end{equation}

% Relations of the same type are gathered in the same extended relation set. Therefore, we have three extended relation set, namely $RS_{\rightarrow}^{M}$, $RS_{\leftarrow}^{M}$ and $RS_{\parallel}^{M}$.  Now we give the definition of intersection and union operation of these sets.

% \begin{definition}[Intersection and Union Operation of Relation Sets]\label{def:relSetOperations}
% Given two extended relation sets $RS_{h}^{P}$ and $RS_{h}^{Q}$ from two models $P$ and $Q$ with the same $h$. Then we have:
% 		\begin{align}
% 			RS_{h}^{P}\Cap RS_{h}^{Q}= & \sum_{R_{h}^{P}\in RS_{h}^{P},R_{h}^{Q}\in RS_{h}^{Q},{R_{h}^{P}}_{pair}={R_{h}^{Q}}_{pair}}{R_{h}^{P}\cap R_{h}^{Q}}\label{eq:RSintersection}\\
% 			RS_{h}^{P}\Cup RS_{h}^{Q}= & \sum_{R_{h}^{P}\in RS_{h}^{P},R_{h}^{Q}\in RS_{h}^{Q},{R_{h}^{P}}_{pair}={R_{h}^{Q}}_{pair}}{R_{h}^{P}\cup R_{h}^{Q}}\label{eq:RSunion1}\\
% 			& +|\{R_{h}^{P}\in RS_{h}^{P}|\nexists R_{h}^{Q}\in RS_{h}^{Q},s.t.~{R_{h}^{P}}_{pair}={R_{h}^{Q}}_{pair}\}|\label{eq:RSunion2}\\
% 			& +|\{R_{h}^{Q}\in RS_{h}^{Q}|\nexists R_{h}^{P}\in RS_{h}^{P},s.t.~{R_{h}^{P}}_{pair}={R_{h}^{Q}}_{pair}\}|\label{eq:RSunion3}
% 		\end{align}
% \end{definition}

% Finally, we give the similarity measure as follows:

% \begin{definition}[ExRORU Similarity]\label{def:similarity}
% Given two WF-nets $P$ and $Q$, the ExRORU similarity between them is:\\
% \begin{equation}\label{eq:similarity}
% Sim(P,Q)=\sum_{h}\omega_{h}\cdot Sim(RS_{h}^{P},RS_{h}^{Q})
% \end{equation}
% with $h\in\{\rightarrow,\leftarrow,\parallel\}$ and weighting factors $\omega_{h}\in\mathbb{R},0<\omega_{h}<1$ such that $\sum\limits_{h}\omega_{h}=1$. The similarity between relation set is:\\
% \begin{equation}\label{eq:setsimilarity}
% Sim(RS_{h}^{P},RS_{h}^{Q})=\frac{|RS_{h}^{P}\Cap RS_{h}^{Q}|}{|RS_{h}^{P}\Cup RS_{h}^{Q}|}
% \end{equation}
% \end{definition}

% \begin{example}\label{ex:similarity}
% For the two models in Figure \ref{fig:similarityExample}, we can capture their extended relations as shown in Table \ref{tab:similarityExample} and \textit{SDA-sets} as follows:
% \begin{displaymath}
% 	\begin{aligned}
% 		\text{(a)} & A\rightarrow_{D}B, B\rightarrow_{D}C, B\rightarrow_{D}D, D\rightarrow_{D}B\\
% 		& B\leftarrow_{D}A, C\leftarrow_{D}B, D\leftarrow_{D}B, B\leftarrow_{D}D\\
% 		\text{(b)} & A\rightarrow_{D}B, A\rightarrow_{D}E, B\rightarrow_{D}C, E\rightarrow_{D}E\\
% 		& B\leftarrow_{D}A, E\leftarrow_{D}A, C\leftarrow_{D}B, C\leftarrow_{D}E
% 	\end{aligned}
% \end{displaymath}
% According to our definitions above, Formula \ref{eq:RSintersection} and \ref{eq:RSunion1} will be applied on the light gray area, while Formula \ref{eq:RSunion2} on the medium gray area and Formula \ref{eq:RSunion3} on the dark gray area. For example, $Sim(RS_{\rightarrow}^{P},RS_{\rightarrow}^{Q})=\frac{6}{9+7+7}=\frac{6}{23}$. Similarly, we have $Sim(RS_{\leftarrow}^{P},RS_{\leftarrow}^{Q})=\frac{6}{23}$ and $Sim(RS_{\parallel}^{P},RS_{\parallel}^{Q})=\frac{9}{23}$. If we use $\omega_{h}=\frac{1}{3}$ as our weighting factors, we get a similarity value between $P$ and $Q$ as $\frac{6/23+6/23+9/23}{3}\approx 0.304$.
% \end{example}

% \begin{figure}[htbp]
% \centering
% \subfigure[Model $P$] {
% 	\centering
% 	\includegraphics[width=0.45\textwidth]{fig_similarity_1}
% 	\label{fig:similarityExampleA}
% }
% \hspace{0.5cm}
% \subfigure[Model $Q$] {
% 	\centering
% 	\includegraphics[width=0.45\textwidth]{fig_similarity_2}
% 	\label{fig:similarityExampleB}
% }
% \caption{Example petri nets $P$ and $Q$ for similarity measure\label{fig:similarityExample}}
% \end{figure}

% \begin{table}[htbp]
% \centering
% \tabcaption{Extended relations of the two models in Figure \ref{fig:similarityExample}\label{tab:similarityExample}}
% \subtable[Extended relation matrix of $P$] {
% 	\begin{tabular}{c|c|c|c|c} \hline
% 		Transitions & $A$ & $B$ & $C$ & $D$\\ \hline
% 		$A$
% 			& \multicolumn{1}{>{\columncolor{gray1}}c|}{$\nrightarrow\nleftarrow\nparallel$}
% 			& \multicolumn{1}{>{\columncolor{gray1}}c|}{$\twoheadrightarrow\nleftarrow\nparallel$}
% 			& \multicolumn{1}{>{\columncolor{gray1}}c|}{$\twoheadrightarrow\nleftarrow\nparallel$}
% 			& \multicolumn{1}{>{\columncolor{gray2}}c|}{$\rightharpoonup\nleftarrow\nparallel$}
% 			\\ \hline
% 		$B$
% 			& \multicolumn{1}{>{\columncolor{gray1}}c|}{$\nrightarrow\leftharpoonup\nparallel$}
% 			& \multicolumn{1}{>{\columncolor{gray1}}c|}{$\rightharpoonup\leftharpoonup\nparallel$}
% 			& \multicolumn{1}{>{\columncolor{gray1}}c|}{$\rightharpoonup\nleftarrow\nparallel$}
% 			& \multicolumn{1}{>{\columncolor{gray2}}c|}{$\rightharpoonup\leftharpoonup\nparallel$}
% 			\\ \hline
% 		$C$
% 			& \multicolumn{1}{>{\columncolor{gray1}}c|}{$\nrightarrow\twoheadleftarrow\nparallel$}
% 			& \multicolumn{1}{>{\columncolor{gray1}}c|}{$\nrightarrow\twoheadleftarrow\nparallel$}
% 			& \multicolumn{1}{>{\columncolor{gray1}}c|}{$\nrightarrow\nleftarrow\nparallel$}
% 			& \multicolumn{1}{>{\columncolor{gray2}}c|}{$\nrightarrow\leftharpoonup\nparallel$}
% 			\\ \hline
% 		$D$
% 			& \multicolumn{1}{>{\columncolor{gray2}}c|}{$\nrightarrow\leftharpoonup\nparallel$}
% 			& \multicolumn{1}{>{\columncolor{gray2}}c|}{$\twoheadrightarrow\twoheadleftarrow\nparallel$}
% 			& \multicolumn{1}{>{\columncolor{gray2}}c|}{$\rightharpoonup\nleftarrow\nparallel$}
% 			& \multicolumn{1}{>{\columncolor{gray2}}c|}{$\rightharpoonup\leftharpoonup\nparallel$}
% 			\\ \hline
% 	\end{tabular}
% }
% \subtable[Extend relation matrix of $Q$] {
% 	\begin{tabular}{c|c|c|c|c} \hline
% 		Transitions & $A$ & $B$ & $C$ & $E$\\ \hline
% 		$A$
% 			& \multicolumn{1}{>{\columncolor{gray1}}c|}{$\nrightarrow\nleftarrow\nparallel$}
% 			& \multicolumn{1}{>{\columncolor{gray1}}c|}{$\rightharpoonup\nleftarrow\nparallel$}
% 			& \multicolumn{1}{>{\columncolor{gray1}}c|}{$\twoheadrightarrow\nleftarrow\nparallel$}
% 			& \multicolumn{1}{>{\columncolor{gray3}}c|}{$\rightharpoonup\nleftarrow\nparallel$}
% 			\\ \hline
% 		$B$
% 			& \multicolumn{1}{>{\columncolor{gray1}}c|}{$\nrightarrow\twoheadleftarrow\nparallel$}
% 			& \multicolumn{1}{>{\columncolor{gray1}}c|}{$\nrightarrow\nleftarrow\nparallel$}
% 			& \multicolumn{1}{>{\columncolor{gray1}}c|}{$\twoheadrightarrow\nleftarrow\nparallel$}
% 			& \multicolumn{1}{>{\columncolor{gray3}}c|}{$\nrightarrow\nleftarrow\nparallel$}
% 			\\ \hline
% 		$C$
% 			& \multicolumn{1}{>{\columncolor{gray1}}c|}{$\nrightarrow\twoheadleftarrow\nparallel$}
% 			& \multicolumn{1}{>{\columncolor{gray1}}c|}{$\nrightarrow\leftharpoonup\nparallel$}
% 			& \multicolumn{1}{>{\columncolor{gray1}}c|}{$\nrightarrow\nleftarrow\nparallel$}
% 			& \multicolumn{1}{>{\columncolor{gray3}}c|}{$\nrightarrow\leftharpoonup\nparallel$}
% 			\\ \hline
% 		$E$
% 			& \multicolumn{1}{>{\columncolor{gray3}}c|}{$\nrightarrow\twoheadleftarrow\nparallel$}
% 			& \multicolumn{1}{>{\columncolor{gray3}}c|}{$\nrightarrow\nleftarrow\nparallel$}
% 			& \multicolumn{1}{>{\columncolor{gray3}}c|}{$\twoheadrightarrow\nleftarrow\nparallel$}
% 			& \multicolumn{1}{>{\columncolor{gray3}}c|}{$\nrightarrow\nleftarrow\nparallel$}
% 			\\ \hline
% 	\end{tabular}
% }
% \end{table}

\section{Experimental Evaluation}\label{sec:experiments}
We have implemented a similarity measure algorithm\footnote{https://github.com/shudiwsh2009/ExRORU} based on the complete prefix unfolding in jbpt\footnote{https://code.google.com/p/jbpt/}. In this section, we present the experimental evaluation of our algorithm from both effectiveness and efficiency perspectives. We evaluate the effectiveness by comparing it with other algorithms and evaluate the efficiency by applying our algorithm on real-life datasets.

\subsection{Effectiveness Comparison}\label{subsec:effectivenessComparison}
Examples in this section together with the models in Figure \ref{fig:sdaExample} show the powerful capability of our algorithm to represent the behaviors of a process model.

\paragraph{Non-free-choice constructs}\label{par:nfc} 
Figure \ref{fig:nfcExampleA} shows a WF-net with non-free-choice constructs \cite{wen2007mining} and Figure \ref{fig:nfcExampleB} shows a similar WF-net that is free choice.

Take TAR algorithm for example, the TAR-sets of them are both $\{\langle A,C\rangle, \langle C,D\rangle$, $\langle B,C\rangle, \langle C,E\rangle\}$. However, it is obvious that their behaviors are not the same, since the trace set $\Omega$ of the model in Figure \ref{fig:nfcExampleA} is $\{\langle A,C,D\rangle,\langle B,C,E\rangle\}$ while the other is $\{\langle A,C,D\rangle,\langle A,C,E\rangle,\langle B,C,D\rangle,\langle B,C,E\rangle\}$.

Actually, there are only two branching processes in the complete prefix unfolding of the model in Figure \ref{fig:nfcExampleA}, i.e., $[A\{C\}D]$ and $[B\{C\}E]$. Therefore, we have $A\twoheadrightarrow D$ and $B\twoheadrightarrow E$. On the other hand, there are four branching processes in the complete prefix unfolding of the model in Figure \ref{fig:nfcExampleB}, i.e., $[ACD],[ACE],[BCD],[BCE]$, indicating that $A\rightharpoonup D$ and $B\rightharpoonup E$. We can see the difference between these two models using our extended relations.

\paragraph{Multi-relation} is another special structure we want to discuss. An example WF-net with multi-relation and its complete prefix unfolding are shown in Figure \ref{fig:exampleMultiRelation}. Transitions $A$ and $D$ are in multi-relation since they can both in causal and concurrency relation. From the unfolding in Figure \ref{fig:multiRelExampleCpu}, we know there are two branching processes, i.e., $\sigma_{1}=[I\{AC,BD\}O]$ and $\sigma_{2}=[I\{A\}E\{D\}O]$. Using the definitions of our extended relations, we have $A\rightharpoonup D$ and $A\Uparrow D$.

\begin{figure}[htbp]
\centering
\subfigure[A WF-net with multi-relation] {
	\centering
	\includegraphics[width=0.46\textwidth]{fig_multi_rel}
	\label{fig:multiRelExample}
}
\subfigure[The complete prefix unfolding of \subref{fig:multiRelExample}] {
	\centering
	\includegraphics[width=0.46\textwidth]{fig_multi_rel_cpu}
	\label{fig:multiRelExampleCpu}
}
\caption{A WF-net with multi-relation and its complete prefix unfolding\label{fig:exampleMultiRelation}}
\end{figure}

Most similarity measure such as BP and TAR cannot recognize the multi-relation in a process model since they categorize transition pair into one type of relation, which causes problems.

\paragraph{Silent transitions} can also influence the behavioral of a model, as mentioned before. In the model pair of Figure \ref{fig:sdaExample} and \ref{fig:silentExamples}, BP fails to recognize the difference between all four cases while TAR fails in Figure \ref{fig:silentExampleB} and \ref{fig:silentExampleC}. On the other hand, our method easily detect the behavioral differentiation between the model pairs with the help of ExRORU and SDA.

\begin{figure}[htbp]
\centering
\subfigure[] {
	\begin{minipage}[b]{0.163\textwidth}
		\centering
		\includegraphics[width=1.0\textwidth]{fig_silent_1}
	\end{minipage}
	\label{fig:silentExampleA}
}
\subfigure[] {
	\begin{minipage}[b]{0.38\textwidth}
		\centering
		\includegraphics[width=1.0\textwidth]{fig_silent_2}
	\end{minipage}
	\label{fig:silentExampleB}
}
\subfigure[] {
	\begin{minipage}[b]{0.38\textwidth}
		\centering
		\includegraphics[width=1.0\textwidth]{fig_silent_3}
	\end{minipage}
	\label{fig:silentExampleC}
}
\caption{Examples with silent transitions that BP and TAR fails the differentiation detection\label{fig:silentExamples}}
\end{figure}

\subsection{Efficiency Comparison}\label{subsec:setting}
Our experimental dataset contains real-life models from several enterprises such as TC, DG and SAP. The basic features of these models are summarized in Table \ref{tab:datasets}.

\begin{table}[htbp]
\centering
\caption{The features of experimental datasets\label{tab:datasets}}
\scriptsize
\begin{tabular}{|c|r|r|r|r|r|r|r|r|r|r|} \hline
	\multicolumn{1}{|c|}{\multirow{2}{*}{Dataset}} & \multicolumn{1}{c|}{\multirow{2}{*}{Size}} & \multicolumn{3}{|c}{Average} & \multicolumn{3}{|c}{Minimum} & \multicolumn{3}{|c|}{Maximum}\\ \cline{3-11}
	\multicolumn{1}{|c|}{} & \multicolumn{1}{c|}{} & \multicolumn{1}{c|}{\#transitions} & \multicolumn{1}{c|}{\#place} & \multicolumn{1}{c|}{\#arcs} & \multicolumn{1}{c|}{\#transitions} & \multicolumn{1}{c|}{\#place} & \multicolumn{1}{c|}{\#arcs} & \multicolumn{1}{c|}{\#transitions} & \multicolumn{1}{c|}{\#place} & \multicolumn{1}{c|}{\#arcs}\\ \hline
	DG & 94 & 8.56 & 8.89 & 17.78 & 1 & 3 & 3 & 34 & 33 & 70 \\ \hline
	TC & 89 & 11.47 & 10.28 & 22.93 & 6 & 5 & 11 & 28 & 29 & 58 \\ \hline
	SAP & 389 & 4.47 & 7.51 & 11.51 & 1 & 2 & 2 & 21 & 31 & 56 \\ \hline
\end{tabular}
\end{table}

As for comparison, TAR and BP algorithms are based on the relations between tasks as we do while PTS and CFS are based on the abstract trace sets which seems closer to reality. Therefore, we conduct experiments using our algorithm together with the four above and compare their efficiency. The average time costs on extracting behavioral features of each process model are shown in Table \ref{tab:efficiency}.

\begin{table}[htbp]
\centering
\caption{Efficiency comparison of five algorithms\label{tab:efficiency}}
\begin{threeparttable}
\begin{tabular}{|c|p{1.5cm}<{\centering}|p{1.5cm}<{\centering}|p{1.5cm}<{\centering}|p{1.5cm}<{\centering}|p{1.5cm}<{\centering}|} \hline
	\diagbox{Dataset}{Cost(ms)}{Algo} & TAR & BP & PTS & CFS & ExRORU\\ \hline
	DG & 0.11 & 2.15 & 0.52 & 0.42 & 10.89\\ \hline
	TC & 0.10 & 2.01 & 0.39 & 0.40 & 15.34\\ \hline
	SAP & 0.05 & 0.37 & 0.06 & 0.07 & 1.43\\ \hline
\end{tabular}
\end{threeparttable}
\end{table}

As we can see from the result, ExRORU is the slowest algorithm among all five due to some reasons. Firstly, models in these datasets are mostly acyclic and lack parallel constructs so that the state explosion problem may not happen in algorithms utilizing reachability graph such as PTS and CFS. Secondly, most of the time consuming in simialrity measure such as CFS are spent in the similarity computation between the behavioral features of models. On the other hand, comparison between two models using ExRORU is extremely fast. Finally, considering that ExRORU contains much more details about relations between tasks than other algorithms, the time consuming of it is acceptable. To our best knowledge, ExRORU is the only existing algorithm that can handle many special cases such as silent transitions, non-free-choice constructs and multi-relation, which shows the outstanding effectiveness of it.

\section{Conclusion}\label{sec:conclusion}
To better describe the relations between the execution of tasks in a process model, we extend Tao Jin's work on refined ordering relations with uncertainty. In our paper, we define three types of extended relations, all of which can be refined into three sub types. We give the computation of our extedned relations based on the unfolding technology. In order to better measure the differentiation between process models, we introduce the notion of sequential direct adjacency. Our algorithm is the only existing algorithm that can handle many special cases and has an acceptable time performance, as shown in the experiments conducted with four real-life model datasets.

Our algorithm is still not fast enough, compared to other mainstream algorithms which also utilize the relations between tasks. On the other hand, we omit the proof of some theorems due to space limitation. We will solve these in the future.

\bibliographystyle{plain}
\bibliography{ref}
\end{document}